\documentclass[11pt, a4paper]{article}

\usepackage[T1]{fontenc}     % Estamos usando pdfLaTeX,
\usepackage[utf8]{inputenc}  % por lo tanto esta preparación
\usepackage[spanish]{babel} % Cambiado a español
\usepackage[left = 0mm, right = 0mm, top = 0mm, bottom = 0mm]{geometry}
\usepackage[stretch = 25, shrink = 25, tracking=true, letterspace=30]{microtype}
\usepackage{graphicx}        % Para insertar imágenes
\usepackage{xcolor}          % Para añadir color al documento
\usepackage{marvosym}        % Provee íconos para los detalles de contacto

\usepackage{enumitem}        % Para redefinir el espaciado en las listas
\setlist{parsep = 0pt, topsep = 0pt, partopsep = 1pt, itemsep = 1pt, leftmargin = 6mm}

\usepackage{FiraSans}        % Cambia esto para usar cualquier fuente, pero mantenlo simple
\renewcommand{\familydefault}{\sfdefault}

\definecolor{cvblue}{HTML}{304263}

%%%%%%% DEFINICIONES DE COMANDOS DE USUARIO %%%%%%%%%%%%%%%%%%%%%%%%%%%
% Estos son los verdaderos caballos de batalla de esta plantilla
\newcommand{\dates}[1]{\hfill\mbox{\textbf{#1}}} % Texto en negrita que no se divide en líneas
\newcommand{\is}{\par\vskip.5ex plus .4ex} % Espaciado de ítem
\newcommand{\smaller}[1]{{\small$\diamond$\ #1}}
\newcommand{\headleft}[1]{\vspace*{3ex}\textsc{\textbf{#1}}\par%
    \vspace*{-1.5ex}\hrulefill\par\vspace*{0.7ex}}
\newcommand{\headright}[1]{\vspace*{2.5ex}\textsc{\Large\color{cvblue}#1}\par%
     \vspace*{-2ex}{\color{cvblue}\hrulefill}\par}
%%%%%%%%%%%%%%%%%%%%%%%%%%%%%%%%%%%%%%%%%%%%%%%%%%%%%%%%%%%%

\usepackage[colorlinks = true, urlcolor = white, linkcolor = white]{hyperref}

\begin{document}

% Definiciones de estilo -- eliminando el espacio innecesario y añadiendo los saltos explícitamente
\setlength{\topskip}{0pt}
\setlength{\parindent}{0pt}
\setlength{\parskip}{0pt}
\setlength{\fboxsep}{0pt}
\pagestyle{empty}
\raggedbottom

\begin{minipage}[t]{0.33\textwidth} %% Columna izquierda -- definición exterior
%  Columna izquierda -- rectángulo oscuro superior
\colorbox{cvblue}{\begin{minipage}[t][5mm][t]{\textwidth}\null\hfill\null\end{minipage}}

\vspace{-.2ex} % Elimina el pequeño hueco
\colorbox{cvblue!90}{\color{white}  %% CAJA IZQUIERDA
\kern0.09\textwidth\relax% Margen izquierdo proporcionado explícitamente
\begin{minipage}[t][293mm][t]{0.82\textwidth}
\raggedright
\vspace*{2.5ex}

\Large \textbf{Osvaldo Estrada} \normalsize

% Centrado sin espaciado vertical extra
\null\hfill\includegraphics[width=0.65\textwidth]{osvaldo.jpg}\hfill\null

\vspace*{0.5ex} % Espacio extra después de la imagen

\headleft{Resumen del Perfil} % Título traducido
Un profundo amor por el aprendizaje, soy
proactivo y organizado, buscando
consistentemente nuevas oportunidades para crecer.
Orientado a resultados, abordo las tareas
con determinación y me esfuerzo por
la excelencia. La comunicación efectiva
es una fortaleza, lo que me permite compartir
ideas claramente y colaborar.


\headleft{Información Personal} % Título traducido
Ubicación: \textbf{México, CDMX} \\[0.5ex] % Traducido
Nacimiento: \textbf{20 de Octubre, 2000} \\[0.5ex] % Traducido
Idiomas: \textbf{Español(Nativo)}, \textbf{Inglés(B2 Intermedio/Intermedio-Alto)} \\[0.5ex] % Traducido
Linkedin: \href{https://www.linkedin.com/in/osvaldo-israel-estrada-sosa-19529919a/}{Click me} \\[0.5ex]
Whatsapp: 55 2491 9990

\headleft{Habilidades} % Título traducido
\begin{itemize}
\item Python, Numpy, Scikit-learn
\item Pyspark, DBT, Azure Databricks
\item Modelado Dimensional, Modelado ER
\item SQLServer, Postgres
\item Flask
\item Git, Github, Bitbucket
\item Linux/UNIX, CLI, Docker
\item Azure Blob Storage, Azure Functions
\item AWS Lambda, AWS S3, AWS EC2, AWS Glue
\item Terraform
\end{itemize}

\headleft{Intereses} % Título traducido
\begin{itemize}
\item Arquitectura Cloud
\item DataOps y CI/CD
\item Modelado Data Vault 2.0
\end{itemize}

\end{minipage}%
\kern0.09\textwidth\relax%%Margen derecho proporcionado explícitamente para estirar la caja de color
}
\end{minipage}% Columna derecha
\hskip2.5em% Margen izquierdo para el área blanca
\begin{minipage}[t]{0.56\textwidth}
\setlength{\parskip}{0.8ex}% Añade espacios entre párrafos; usa \\ para añadir nuevas líneas sin este espacio. Reduce esta cantidad para encajar más datos verticalmente

\vspace{2ex}

\headright{Experiencia} % Título traducido

\textsc{Ingeniero de Datos}\\ % Título traducido
\textit{en Exploration \& Discovery Technologies}\\ % 'at' traducido a 'en'
\dates{Noviembre 2023 - }\\ % Mes traducido
Ingeniero de Datos con trayectoria comprobada en la optimización de pipelines de datos y la mejora de la eficiencia operativa para un sistema de seguimiento de flotas. Experimentado en el aprovechamiento de tecnologías como Python, SQL, Docker, DBT, Databricks y PySpark. Competente en modelado de datos, procesos ETL y prácticas CI/CD, desde el desarrollo hasta la implementación, utilizando tecnologías cloud como Azure Blob Storage, Azure Functions, AWS Lambda, AWS S3, AWS EC2, AWS Glue. % Texto traducido

\vspace{0.2cm}
\textsc{Científico de Datos}\\ % Título traducido
\textit{en Insaite}\\ % 'at' traducido a 'en'
\dates{Julio 2022 - Febrero 2023}\\ % Meses traducidos
He trabajado en pipelines de datos completos para predicción de abandono (churn), informes médicos, clasificación de texto y paneles de control (dashboards) para la toma de decisiones.
Involucrado en la ingesta de datos, pasando por el análisis exploratorio de datos, utilizando manipulaciones avanzadas de ingeniería de datos, almacenamiento en almacenes de datos (warehouses) y bases de datos relacionales, implementando finalmente KPIs útiles y modelos de aprendizaje automático/aprendizaje profundo (machine learning/deep learning). % Texto traducido

\vspace{0.2cm}
\textsc{Ingeniero de Datos}\\ % Título traducido
\textit{en Instituto de Investigaciones en Matemáticas Aplicadas y Sistemas}\\ % 'at' traducido a 'en'
\dates{Enero 2021 - Diciembre 2021}\\ % Meses traducidos
Diseño e implementación del pipeline completo para extraer, limpiar y analizar videos de Tiktok, con la ayuda de técnicas de web scraping. Gestión, administración de sistemas linux y SGBDR (RDBMS), poblándolos con reconocimiento de objetos a partir de una red neuronal convolucional sobre GPUs y finalmente predecir escenas de riesgo de COVID-19 según la OMS con el objetivo de disminuir la infección. % Texto traducido

\headright{Educación} % Título traducido

\smaller{Licenciatura en Matemáticas Aplicadas y Ciencias de la Computación}\\ % Título traducido
\smaller{Universidad Nacional Autónoma de México}. \dates{2019-2022}

\headright{Certificaciones} % Título traducido

\smaller{AWS Certified Cloud Practitioner} % Dejo los nombres de las certificaciones en inglés
\is
\smaller{Databricks Certified Associate Developer for Apache Spark 3.0}
\is
\smaller{Databricks Certified Data Engineer Associate}
\is
\smaller{Scrum Master Certified}

\end{minipage}

\end{document}\documentclass[11pt, a4paper]{article}

\usepackage[T1]{fontenc}     % Estamos usando pdfLaTeX,
\usepackage[utf8]{inputenc}  % por lo tanto esta preparación
\usepackage[spanish]{babel} % Cambiado a español
\usepackage[left = 0mm, right = 0mm, top = 0mm, bottom = 0mm]{geometry}
\usepackage[stretch = 25, shrink = 25, tracking=true, letterspace=30]{microtype}
\usepackage{graphicx}        % Para insertar imágenes
\usepackage{xcolor}          % Para añadir color al documento
\usepackage{marvosym}        % Provee íconos para los detalles de contacto

\usepackage{enumitem}        % Para redefinir el espaciado en las listas
\setlist{parsep = 0pt, topsep = 0pt, partopsep = 1pt, itemsep = 1pt, leftmargin = 6mm}

\usepackage{FiraSans}        % Cambia esto para usar cualquier fuente, pero mantenlo simple
\renewcommand{\familydefault}{\sfdefault}

\definecolor{cvblue}{HTML}{304263}

%%%%%%% DEFINICIONES DE COMANDOS DE USUARIO %%%%%%%%%%%%%%%%%%%%%%%%%%%
% Estos son los verdaderos caballos de batalla de esta plantilla
\newcommand{\dates}[1]{\hfill\mbox{\textbf{#1}}} % Texto en negrita que no se divide en líneas
\newcommand{\is}{\par\vskip.5ex plus .4ex} % Espaciado de ítem
\newcommand{\smaller}[1]{{\small$\diamond$\ #1}}
\newcommand{\headleft}[1]{\vspace*{3ex}\textsc{\textbf{#1}}\par%
    \vspace*{-1.5ex}\hrulefill\par\vspace*{0.7ex}}
\newcommand{\headright}[1]{\vspace*{2.5ex}\textsc{\Large\color{cvblue}#1}\par%
     \vspace*{-2ex}{\color{cvblue}\hrulefill}\par}
%%%%%%%%%%%%%%%%%%%%%%%%%%%%%%%%%%%%%%%%%%%%%%%%%%%%%%%%%%%%

\usepackage[colorlinks = true, urlcolor = white, linkcolor = white]{hyperref}

\begin{document}

% Definiciones de estilo -- eliminando el espacio innecesario y añadiendo los saltos explícitamente
\setlength{\topskip}{0pt}
\setlength{\parindent}{0pt}
\setlength{\parskip}{0pt}
\setlength{\fboxsep}{0pt}
\pagestyle{empty}
\raggedbottom

\begin{minipage}[t]{0.33\textwidth} %% Columna izquierda -- definición exterior
%  Columna izquierda -- rectángulo oscuro superior
\colorbox{cvblue}{\begin{minipage}[t][5mm][t]{\textwidth}\null\hfill\null\end{minipage}}

\vspace{-.2ex} % Elimina el pequeño hueco
\colorbox{cvblue!90}{\color{white}  %% CAJA IZQUIERDA
\kern0.09\textwidth\relax% Margen izquierdo proporcionado explícitamente
\begin{minipage}[t][293mm][t]{0.82\textwidth}
\raggedright
\vspace*{2.5ex}

\Large \textbf{Osvaldo Estrada} \normalsize

% Centrado sin espaciado vertical extra
\null\hfill\includegraphics[width=0.65\textwidth]{osvaldo.jpg}\hfill\null

\vspace*{0.5ex} % Espacio extra después de la imagen

\headleft{Resumen del Perfil} % Título traducido
Un profundo amor por el aprendizaje, soy
proactivo y organizado, buscando
consistentemente nuevas oportunidades para crecer.
Orientado a resultados, abordo las tareas
con determinación y me esfuerzo por
la excelencia. La comunicación efectiva
es una fortaleza, lo que me permite compartir
ideas claramente y colaborar.


\headleft{Información Personal} % Título traducido
Ubicación: \textbf{México, CDMX} \\[0.5ex] % Traducido
Nacimiento: \textbf{20 de Octubre, 2000} \\[0.5ex] % Traducido
Idiomas: \textbf{Español(Nativo)}, \textbf{Inglés(B2 Intermedio/Intermedio-Alto)} \\[0.5ex] % Traducido
Linkedin: \href{https://www.linkedin.com/in/osvaldo-israel-estrada-sosa-19529919a/}{Click me} \\[0.5ex]
Whatsapp: 55 2491 9990

\headleft{Habilidades} % Título traducido
\begin{itemize}
\item Python, Numpy, Scikit-learn
\item Pyspark, DBT, Azure Databricks
\item Modelado Dimensional, Modelado ER
\item SQLServer, Postgres
\item Flask
\item Git, Github, Bitbucket
\item Linux/UNIX, CLI, Docker
\item Azure Blob Storage, Azure Functions
\item AWS Lambda, AWS S3, AWS EC2, AWS Glue
\item Terraform
\end{itemize}

\headleft{Intereses} % Título traducido
\begin{itemize}
\item Arquitectura Cloud
\item DataOps y CI/CD
\item Modelado Data Vault 2.0
\end{itemize}

\end{minipage}%
\kern0.09\textwidth\relax%%Margen derecho proporcionado explícitamente para estirar la caja de color
}
\end{minipage}% Columna derecha
\hskip2.5em% Margen izquierdo para el área blanca
\begin{minipage}[t]{0.56\textwidth}
\setlength{\parskip}{0.8ex}% Añade espacios entre párrafos; usa \\ para añadir nuevas líneas sin este espacio. Reduce esta cantidad para encajar más datos verticalmente

\vspace{2ex}

\headright{Experiencia} % Título traducido

\textsc{Ingeniero de Datos}\\ % Título traducido
\textit{en Exploration \& Discovery Technologies}\\ % 'at' traducido a 'en'
\dates{Noviembre 2023 - }\\ % Mes traducido
Ingeniero de Datos con trayectoria comprobada en la optimización de pipelines de datos y la mejora de la eficiencia operativa para un sistema de seguimiento de flotas. Experimentado en el aprovechamiento de tecnologías como Python, SQL, Docker, DBT, Databricks y PySpark. Competente en modelado de datos, procesos ETL y prácticas CI/CD, desde el desarrollo hasta la implementación, utilizando tecnologías cloud como Azure Blob Storage, Azure Functions, AWS Lambda, AWS S3, AWS EC2, AWS Glue. % Texto traducido

\vspace{0.2cm}
\textsc{Científico de Datos}\\ % Título traducido
\textit{en Insaite}\\ % 'at' traducido a 'en'
\dates{Julio 2022 - Febrero 2023}\\ % Meses traducidos
He trabajado en pipelines de datos completos para predicción de abandono (churn), informes médicos, clasificación de texto y paneles de control (dashboards) para la toma de decisiones.
Involucrado en la ingesta de datos, pasando por el análisis exploratorio de datos, utilizando manipulaciones avanzadas de ingeniería de datos, almacenamiento en almacenes de datos (warehouses) y bases de datos relacionales, implementando finalmente KPIs útiles y modelos de aprendizaje automático/aprendizaje profundo (machine learning/deep learning). % Texto traducido

\vspace{0.2cm}
\textsc{Ingeniero de Datos}\\ % Título traducido
\textit{en Instituto de Investigaciones en Matemáticas Aplicadas y Sistemas}\\ % 'at' traducido a 'en'
\dates{Enero 2021 - Diciembre 2021}\\ % Meses traducidos
Diseño e implementación del pipeline completo para extraer, limpiar y analizar videos de Tiktok, con la ayuda de técnicas de web scraping. Gestión, administración de sistemas linux y SGBDR (RDBMS), poblándolos con reconocimiento de objetos a partir de una red neuronal convolucional sobre GPUs y finalmente predecir escenas de riesgo de COVID-19 según la OMS con el objetivo de disminuir la infección. % Texto traducido

\headright{Educación} % Título traducido

\smaller{Licenciatura en Matemáticas Aplicadas y Ciencias de la Computación}\\ % Título traducido
\smaller{Universidad Nacional Autónoma de México}. \dates{2019-2022}

\headright{Certificaciones} % Título traducido

\smaller{AWS Certified Cloud Practitioner} % Dejo los nombres de las certificaciones en inglés
\is
\smaller{Databricks Certified Associate Developer for Apache Spark 3.0}
\is
\smaller{Databricks Certified Data Engineer Associate}
\is
\smaller{Scrum Master Certified}

\end{minipage}

\end{document}

