\documentclass[11pt, a4paper]{article}

\usepackage[T1]{fontenc}     % We are using pdfLaTeX,
\usepackage[utf8]{inputenc}  % hence this preparation
\usepackage[spanish]{babel} % CAMBIADO a spanish
\usepackage[left = 0mm, right = 0mm, top = 0mm, bottom = 0mm]{geometry}
\usepackage[stretch = 25, shrink = 25, tracking=true, letterspace=30]{microtype}
\usepackage{graphicx}        % To insert pictures
\usepackage{xcolor}          % To add colour to the document
\usepackage{marvosym}        % Provides icons for the contact details

\usepackage{enumitem}        % To redefine spacing in lists
\setlist{parsep = 0pt, topsep = 0pt, partopsep = 1pt, itemsep = 1pt, leftmargin = 6mm}

\usepackage{FiraSans}        % Change this to use any font, but keep it simple
\renewcommand{\familydefault}{\sfdefault}

\definecolor{cvblue}{HTML}{304263}

%%%%%%% USER COMMAND DEFINITIONS %%%%%%%%%%%%%%%%%%%%%%%%%%%
% These are the real workhorses of this template
\newcommand{\dates}[1]{\hfill\mbox{\textbf{#1}}} % Bold stuff that doesn’t got broken into lines
\newcommand{\is}{\par\vskip.5ex plus .4ex} % Item spacing
\newcommand{\smaller}[1]{{\small$\diamond$\ #1}}
\newcommand{\headleft}[1]{\vspace*{3ex}\textsc{\textbf{#1}}\par%
    \vspace*{-1.5ex}\hrulefill\par\vspace*{0.7ex}}
\newcommand{\headright}[1]{\vspace*{2.5ex}\textsc{\Large\color{cvblue}#1}\par%
     \vspace*{-2ex}{\color{cvblue}\hrulefill}\par}
%%%%%%%%%%%%%%%%%%%%%%%%%%%%%%%%%%%%%%%%%%%%%%%%%%%%%%%%%%%%

\usepackage[colorlinks = true, urlcolor = white, linkcolor = white]{hyperref}

\begin{document}

% Style definitions -- killing the unnecessary space and adding the skips explicitly
\setlength{\topskip}{0pt}
\setlength{\parindent}{0pt}
\setlength{\parskip}{0pt}
\setlength{\fboxsep}{0pt}
\pagestyle{empty}
\raggedbottom

\begin{minipage}[t]{0.33\textwidth} %% Left column -- outer definition
%  Left column -- top dark rectangle
\colorbox{cvblue}{\begin{minipage}[t][5mm][t]{\textwidth}\null\hfill\null\end{minipage}}

\vspace{-.2ex} % Eliminates the small gap
\colorbox{cvblue!90}{\color{white}  %% LEFT BOX
\kern0.09\textwidth\relax% Left margin provided explicitly
\begin{minipage}[t][293mm][t]{0.82\textwidth}
\raggedright
\vspace*{2.5ex}

\Large \textbf{Osvaldo Estrada} \normalsize

% Centering without extra vertical spacing
\null\hfill\includegraphics[width=0.65\textwidth]{osvaldo.jpg}\hfill\null

\vspace*{0.5ex} % Extra space after the picture

\headleft{Resumen del Perfil} % TRADUCIDO
Un profundo amor por el aprendizaje, soy
proactivo y organizado, buscando
consistentemente nuevas oportunidades para crecer.
Orientado a resultados, abordando las tareas
con determinación y esforzándome por
la excelencia. La comunicación efectiva
es una fortaleza, permitiéndome compartir
ideas claramente y colaborar.


\headleft{Información Personal} % TRADUCIDO
Ubicación: \textbf{México, CDMX} \\[0.5ex] % TRADUCIDO
Nacimiento: \textbf{20 de octubre de 2000} \\[0.5ex] % TRADUCIDO
Idiomas: \textbf{Español (Nativo)}, \textbf{Inglés (B2 Intermedio/Intermedio-Alto)} \\[0.5ex] % TRADUCIDO
Linkedin: \href{https://www.linkedin.com/in/osvaldo-israel-estrada-sosa-19529919a/}{Haz clic aquí} \\[0.5ex] % TRADUCIDO
Whatsapp: 55 2491 9990

\headleft{Habilidades} % TRADUCIDO
\begin{itemize}
\item Python, Numpy, Scikit-learn
\item Pyspark, DBT, Azure Databricks
\item Modelado Dimensional, Modelado ER
\item SQLServer, Postgres
\item Flask
\item Git, Github, Bitbucket
\item Linux/UNIX, CLI, Docker
\item Azure Blob Storage, Azure Functions
\item AWS Lambda, AWS S3, AWS EC2, AWS Glue
\item Terraform
\end{itemize}

\headleft{Intereses} % TRADUCIDO
\begin{itemize}
\item Arquitectura Cloud
\item DataOps y CI/CD
\item Modelado Data Vault 2.0
\end{itemize}

\end{minipage}%
\kern0.09\textwidth\relax%%Right margin provided explicitly to stretch the colourbox
}
\end{minipage}% Right column
\hskip2.5em% Left margin for the white area
\begin{minipage}[t]{0.56\textwidth}
\setlength{\parskip}{0.8ex}% Adds spaces between paragraphs; use \\ to add new lines without this space. Shrink this amount to fit more data vertically

\vspace{2ex}

\headright{Experiencia} % TRADUCIDO

\textsc{Ingeniero de Datos}\\ % TRADUCIDO
\textit{en Exploration \& Discovery Technologies}\\ % TRADUCIDO
\dates{Noviembre 2023 - }\\ % TRADUCIDO
Ingeniero de Datos con un historial comprobado en la optimización de pipelines de datos y la mejora de la eficiencia operativa para un sistema de rastreo de flotas. Con experiencia en el aprovechamiento de tecnologías como Python, SQL, Docker, DBT, Databricks y PySpark. Competente en modelado de datos, procesos ETL y prácticas CI/CD, desde el desarrollo hasta la implementación utilizando tecnologías cloud como Azure Blob Storage, Azure Functions, AWS Lambda, AWS S3, AWS EC2, AWS Glue. % TRADUCIDO

\vspace{0.2cm}
\textsc{Científico de Datos}\\ % TRADUCIDO
\textit{en Insaite}\\
\dates{Julio 2022 - Febrero 2023}\\
He trabajado en pipelines de datos completos para predicción de abandono (churn), reportes médicos, clasificación de texto y paneles de control para la toma de decisiones.
Involucrado en la ingesta de datos, pasando por el análisis exploratorio de datos, utilizando manipulaciones avanzadas de ingeniería de datos, almacenamiento en almacenes de datos y bases de datos relacionales, finalmente implementando KPIs útiles y modelos de machine learning/deep learning. % TRADUCIDO

\vspace{0.2cm}
\textsc{Ingeniero de Datos}\\ % TRADUCIDO
\textit{en Instituto de Investigaciones en Matemáticas Aplicadas y Sistemas}\\ % TRADUCIDO
\dates{Enero 2021 - Diciembre 2021}\\ % TRADUCIDO
Diseño e implementación del pipeline completo para extraer, limpiar y analizar videos de Tiktok, con ayuda de técnicas de web scraping. Gestión, administración de sistemas linux y SGBDR, poblarlos con reconocimiento de objetos a partir de una red neuronal convolucional sobre GPUs y finalmente predecir escenas de riesgo de COVID-19 según la OMS con el objetivo de disminuir la infección. % TRADUCIDO

\headright{Educación} % TRADUCIDO

\smaller{Licenciatura en Ingeniería en Matemáticas Aplicadas y Computación}\\ % TRADUCIDO
\smaller{Universidad Nacional Autónoma de México}. \dates{2019-2022}

\headright{Certificaciones} % TRADUCIDO

\smaller{AWS Cloud Practitioner Certificado} % TRADUCIDO
\is
\smaller{Databricks Certified Associate Developer para Apache Spark 3.0} % TRADUCIDO
\is
\smaller{Databricks Certified Data Engineer Associate} % TRADUCIDO
\is
\smaller{Scrum Master Certificado} % TRADUCIDO

\end{minipage}

\end{document}

